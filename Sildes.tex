%!TEX program = xelatex
\documentclass[10pt, compress]{beamer}
\usetheme[titleprogressbar]{m}
\usepackage{booktabs}
\usepackage[italian]{babel}
\usepackage[latin1]{inputenc}
\usepackage[scale=2]{ccicons}
\usepackage[utf8]{inputenc}
\usepackage{enumitem}
\usepackage{multicol}
\usepackage{pst-node}
\usepackage[inline]{enumitem}
\usepackage{tikz}
\usepackage{biblatex} %Imports biblatex package
\addbibresource{sample.bib} %Import the bibliography file
\usepackage{tikz}
\usetikzlibrary{matrix}
\usepackage{tikz-cd}
\usepackage{pgfplots}
\usepackage{pst-node}
%\uspackage{auto-pst-pdf}
\usepackage{tikz-cd} 
\usepackage{graphicx, animate}
\usepackage{graphicx}
\pgfplotsset{%
    ,compat=1.12
    ,every axis x label/.style={at={(current axis.right of origin)},anchor=north west}
    ,every axis y label/.style={at={(current axis.above origin)},anchor=north east}
    }
\newlist{inlineroman}{enumerate*}{1}
\setlist[inlineroman]{itemjoin*={{, and }},afterlabel=~,label=\Roman*.}

\newcommand{\inlinerom}[1]{
\begin{inlineroman}
#1
\end{inlineroman}
}

\newcommand{\ctikz}[1]{
\begin{center}
\begin{tikzpicture}
\node{#1}
\end{tikzpicture}
\end{center}
}

\newcommand{\ctikzo}[1]{
\begin{center}
\begin{tikzpicture}
\node[opacity=0]{#1}
\end{tikzpicture}
\end{center}
}

\newcommand{\tikzz}[1]{
\begin{tikzpicture}
\node{#1}
\end{tikzpicture}
}

\newcommand{\tikzo}[1]{
\begin{tikzpicture}
\node[opacity=0]{#1}
\end{tikzpicture}
}

\newcommand{\ctikzi}[1]{
\begin{center}
\begin{tikzpicture}
\node[opacity=0.2]{#1}
\end{tikzpicture}
\end{center}
}

\newcommand{\tikzi}[1]{
\begin{tikzpicture}
\node[opacity=0.2]{#1}
\end{tikzpicture}
}

\newcommand{\semitransp}[2][1]{\textcolor{#1}{#2}}

\newcommand{\comment}[1]{}
\usepgfplotslibrary{dateplot}

\title{A brief wealth distribution analysis}
\subtitle{}
\author{Andrea Scalenghe}
\institute{Equazioni della Fisica Matematica}
%\date{22 Luglio 2022}
\begin{document}

\maketitle

\begin{frame}{Framework}
    
    \begin{center}
        Ideal Gas $\rightarrow$ Market \quad - \quad Molecule $\rightarrow$ Economic Agent
    \end{center}
    \newline
    \begin{center}
        Microscopic Interaction $\rightarrow$ Microeconomic description
    \end{center}

    \includegraphics[scale=0.5]{gases.png}
    \centering

\end{frame}

\begin{frame}{Microeconomic description}
    Let us consider $v,w\in \mathbb{R}_+$ the wealth of two agents. After their interaction the updated values will be $v^{*},w^{*}$:
    \[\begin{cases}
v^{*}=p_1v+q_1w \\
w^{*}=q_2v+p_2w
\end{cases}\]
Where the \textit{interaction coefficients} $p_i$ and $q_i$ are random variables that respectively represent for the agent $i$ the fraction of her wealth that she keeps for herself and the fraction of $j$'s wealth traded.
\end{frame}

\begin{frame}{From micro to macro}
    Let $X_t,Y_t$ be random variables which describe the number of agents with wealth $v,w$ at time t:

    \[X^{*}_t=p_1X_t+q_1Y_t \quad Y^{*}_t=q_2X_t+p_2Y_t\]

    Both $X_t,Y_t$ have density function $f(v,t)$. 
\end{frame}

\begin{frame}{From micro to macro}
    \begin{itemize}
    
    \item Let us suppose that in a time interval $\Delta t$ the probablity of an interaction to happen is directly proportional to $\Delta t$.
    \item Let $T$ be a random variable such that:

    \[\mathbb{P}(T=1)=\Delta t \quad \mathbb{P}(T=0)=1-\Delta t\]

    Where $\{T=1\}$ is the event in which the interaction happens.
    
    \item Then:

    \[X(t+\Delta t)=TX^{*}(t)+(1-T)X(t)\]

    We assume $X$ and $T$ to be independent.
    \end{itemize}
\end{frame}

\begin{frame}{From micro to macro}
    Let $\varphi\in C^{\infty}(\mathbb{R})$ be a function that represents an observable quantity and let us evaluate the mean value of $\varphi(X(t))$ which we will denote $\langle\varphi(X(t))\rangle$:

\[\langle\varphi(X(t+\Delta t))\rangle=\langle\varphi(TX^{*}(t)+(1-T)X(t))\rangle=\Delta t\langle\varphi(X^{*}(t))\rangle+(1-\Delta t)\langle\varphi(X(t))\rangle\]
Which leads us to:

\[\frac{\langle\varphi(X(t+\Delta t))\rangle-\langle\varphi(X(t))\rangle}{\Delta t}=\langle\varphi(X^{*}(t))\rangle-\langle\varphi(X(t))\rangle\]

Passing to the limit:

    \[\frac{d}{dt}\langle\varphi(X(t))\rangle=\langle\varphi(X^{*}(t))-\varphi(X(t))\rangle\]
\end{frame}

\begin{frame}{From micro to macro}
    Same computations hold for $Y_t$ and therefore we get:

    \[2\frac{d}{dt}\langle\varphi(X_t)\rangle=\langle\varphi(X^{*}(t))+\varphi(Y^{*}(t))-\varphi(X(t))-\varphi(Y(t))\rangle\]

    We now assume independence between $X$ and $Y$ and get:

    \[ \frac{d}{dt}\int_{\mathbb{R}}\varphi(v)f(v,t)dv=\frac{1}{2}\left\langle\int_{\mathbb{R}^2}(\varphi(v^{*})+\varphi(w^{*})-\varphi(v)-\varphi(w))f(v,t)f(w,t)dvdw\right\rangle \]
\end{frame}

\begin{frame}{A different model}
    We consider the following model:

    \[v^{*} = v + P_E(v)z - P(v)v + Q(v)\eta\]

    Where $z$ is the amount of wealth absorbed by the agent (and $M_E$ its mean value) from the environment and $P_E(v)$ is the intensity of the variation of the trait $v$ due to the environment. $Q$ is the variation of the wealth due to random effects $\eta$. We get:

    \[\frac{d}{dt}\int_{I}f(v,t)\varphi(v) \,dv = \left\langle\int_{I\times I}(\varphi(v^{*})-\varphi(v))f(v,t)\mathcal{E}(z) \,dv\,dz\right\rangle\]
\end{frame}

\begin{frame}{From Boltzmann to Fokker-Plank}
    Under suitable hypotheses and through a perturbation of the momenta we end up with:

\hspace{5mm}

    \[\frac{\partial h}{\partial\tau}=\frac{\lambda}{2}\frac{\partial^2}{\partial v^2}(Q(v)^2h)+\frac{\partial}{\partial v}((P(v)v-P_E(v)M_E)h)\]

\hspace{5mm}

    In order to get this PDE we impose the conservation of mass and of the mean value.
\end{frame}



\begin{frame}{Reality check}
    Pareto-Mandelbrot theory states that given $f(w)$ number of agents with wealth $w$ then for sufficiently large $w$ holds:
    
    \hspace{5mm}
    
    \[\int_w^{+\infty}f(v) \,dv \sim w^{-\mu}\]

\hspace{5mm}

    Where $\mu\in(1,2.5)$ is the Pareto index. We will check if the stationary distribution satisfies this condition.
\end{frame}

\begin{frame}{A significant model}
    Until now we presented the general model, let us study the following one:

\begin{equation*}\label{Model}
\begin{cases}
v^{*}=\left(1-\gamma+\eta_1\right)v+\gamma w \\
w^{*}=\left(1-\gamma+\eta_2\right)w+\gamma v
\end{cases}
\end{equation*}

Where $\gamma\in(0,1)$ is the \textit{risk propensity} while $\eta_i$ are random variables, independent of $v$ and $w$, that represent the system's contribution to the trade for each agent.

\end{frame}

\begin{frame}{A significant model}
    In order to get conservation of the average wealth in the model we have to impose that the random parameters are centered, i.e. $\langle\eta_i\rangle=0, i=1,2$, in fact:
    
    \[\langle v^{*}+w^{*}\rangle=(1+\langle\eta_1\rangle)v +(1+\langle\eta_2\rangle)w\]
    
    And the coefficients become:

\[P_E(v) = \gamma \quad P(v) = -\gamma \quad Q(v) = v\]

While the PDE is:

\[\frac{\partial h}{\partial\tau}=\frac{\lambda}{2}\frac{\partial^2}{\partial v^2}(v^2h)+\gamma\frac{\partial}{\partial v}((v-1)h)\]
\end{frame}


\begin{frame}{Stationary solution}
    Let's solve the related stationary equation. We can rewrite the latter as follows:

\[\partial_v\left[\left(\gamma(v-1)+\frac{\lambda}{2}v\right)h+\frac{\lambda}{2}v\partial_v(vh)\right]=0\]

Then by integrating per parts over $[0,v]$ we get:

\[\left(\gamma(v-1)+\frac{\lambda}{2}v\right)h+\frac{\lambda}{2}v\partial_v(vh)=0\]

Rewriting ($\partial_vh=\dot{h}$):

\[\dot{h}+2\left(\frac{\lambda+\gamma}{\lambda v}-\frac{\gamma}{\lambda v^2}\right)h=0\]
\end{frame}

\begin{frame}{Stationary solution}
    Then we get:

    \[h(v) = c\frac{exp\left\{-\frac{\mu-1}{v}\right\}}{v^{\mu+1}}\]

    Where $\mu=1+2\frac{\gamma}{\lambda}>1$ and $c\in\mathbb{R}$ is such that:

    \[\int_{\mathbb{R}}c\frac{e^{-\frac{\mu-1}{v}}}{v^{\mu+1}} \,dv=1 \Rightarrow c=\frac{\Gamma(\mu)}{(\mu-1)^{\mu}}\]

    Finally:

    \[h_{\infty}(v)=\frac{(\mu-1)^{\mu}}{\Gamma(\mu)}\frac{exp\left\{-\frac{\mu-1}{v}\right\}}{v^{\mu+1}}\]
    
\end{frame}

\begin{frame}{Dealing with the PDE}
    By expanding the derivatives we get:

\begin{equation*}\label{eq1.1}
    \frac{\lambda}{2}v^2\partial_{vv}h+\left((2\lambda+\gamma)v-\gamma)\right)\partial_vh+(\lambda+\gamma)h=\partial_t h
\end{equation*}

We would like to determine the solution of the PDE through classical heat equation theory.

Freeing the highest order derivative from the independent variable we get:

\[\partial_{vv}h+\frac{2}{\lambda v^2}\left((2\lambda+\gamma)v-\gamma)\right)\partial_vh+\frac{2(\lambda+\gamma)}{\lambda v^2}h\]


\end{frame}

\begin{frame}{Dealing with the PDE}
  We have no hope in any change of variables that allows us to get a heat equation. Indeed holds the following.

\begin{theorem}
Let $p,q\in C(\mathbb{R}_+)$ and the ODE:
\begin{equation*}\label{ode1}\ddot{y}+p\dot{y}+qy=0\end{equation*}
There exists a change of variable such that the equation \ref{ode1} has costant coefficients if and only if $\frac{(\dot{q}+2pq)}{q^{3/2}}$ is constant.
\end{theorem}
  
\end{frame}

\begin{frame}{A simpler model}
    Let us suppose that in our model the stochastic component is null, i.e. $\eta_i\equiv0$. This assumption describes a closed economic system, in which the two agents interact without any external contribution. Then we end up with:

\[\begin{cases}
v^{*}=\left(1-\gamma\right)v+\gamma w \\
w^{*}=\left(1-\gamma\right)w+\gamma v
\end{cases}\]


\end{frame}

\begin{frame}{A simpler model}
    The corresponding Fokker-Planck equation is:

\begin{equation*}
    \frac{\partial h}{\partial t}=\gamma\frac{\partial}{\partial v}((v-M_E)h)
\end{equation*}

Which can be rewritten as:
\begin{equation*}\label{eq2}
    \partial_th-\gamma (v-M_E)\partial_v h=\gamma h
\end{equation*}

In which we recognize a semilinear PDE. 
\end{frame}

\begin{frame}{A simpler model}
    By the method of characteristics we solve the PDE. We have to solve the following Cauchy's problem:
\[\begin{cases}
\dot{\alpha}=-\gamma (\alpha - M_E) \\
\dot{\beta}=\gamma \beta
\end{cases},\quad \begin{cases}
\alpha(0)=r \\
\beta(0)=h_0(r)
\end{cases}\]

Where we set $h_0$ to be the initial distribution of wealth. It is solved by:
    \[\alpha(t)=(r-M_E)e^{-\gamma t}+M_E,\quad \beta(t)=h_0(r)e^{\gamma t}\]
\end{frame}

\begin{frame}{A simpler model}

    Then by inverting $v=\alpha(t,r)$ we get:

\hspace{5mm}

\[r=(v-M_E)e^{\gamma t}+M_E\]

And then:

\hspace{5mm}

\[h(v,t)=e^{\gamma t}h_0\left((v-M_E)e^{\gamma t}+M_E\right)\]

\end{frame}

\begin{frame}{The uniform distribution}
    If we assume that wealth is uniformly distributed over a closed interval $[0,2M_E]$ then while time passes the distribution collapses into a Dirac delta centered in the average $M_E$. For $M_E=5$ and $\gamma=0.2$ we get:

\begin{center}
\scalebox{0.7}{
\begin{tikzpicture}
\begin{axis}[
    axis lines = left,
    xlabel = \(v\),
    ylabel = {\(h(v)\)}, 
    ymin=0,
    ymax=20,
]

\addplot[
domain=0:10,
smaples=100,
color=grey
] coordinates {(5,0)(5,20)};
%Below the red parabola is defined
\addplot [
    domain=0:10,
    samples=100, 
    color=red,
]
{5};
\addlegendentry{$t=0$}
%Here the blue parabola is defined
\addplot [
    domain=0:10, 
    samples=100, 
    color=blue,
    ]
    {6};
\addlegendentry{$t=1$}

\addplot [
domain=1:9,
samples=100,
color=green,
]{7};
\addlegendentry{$t=2$}

\addplot [
domain=2:8,
samples=100,
color=yellow,
]{9};
\addlegendentry{$t=3$}

\addplot [
domain=3:7,
samples=100,
color=orange,
]{11};
\addlegendentry{$t=4$}

\addplot [
domain=4:6,
samples=100,
color=lightblue,
]{14};
\addlegendentry{$t=5$}

\addplot [
domain=4.5:5.5,
samples=100,
color=brown,
]{17};
\addlegendentry{$t=6$}


\addplot [
domain=4.8:5.2,
samples=100,
color=purple,
]{20};
\addlegendentry{$t=7$}

\end{axis}
\end{tikzpicture}}
\end{center}

\end{frame}

\begin{frame}{Bibliography}
    \nocite{Furioli2017FokkerPlanckEI}
    \nocite{second}
    \nocite{Cordier_2005}
   
   \printbibliography
\end{frame}

\end{document}